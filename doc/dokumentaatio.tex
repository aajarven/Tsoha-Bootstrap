\documentclass[12pt,a4paper,titlepage]{article}
\usepackage[utf8]{inputenc}
\usepackage[finnish]{babel}
\usepackage{setspace}
\usepackage{subcaption}
\usepackage{fancyhdr}
\usepackage[top=1in, bottom=1in, left=1in, right=1in]{geometry}
\usepackage{float}
\usepackage{pdfpages}

\usepackage{hyperref} % lisääthän omat pakettisi ENNEN hyperref'iä
\hypersetup{pdfborder={0 0 0}}
\onehalfspacing
\cfoot{}
\rhead{\thepage}
% asettaa nyk. kappaleen nimen vasempaan ylänurkkaan, saa poistaa jos haluaa
\lhead{\leftmark}

\title{Tsoha\\ Kurssikysely \vspace{0.5em}}
\author{Anni Järvenpää}
\date{\today}

\begin{document}
\maketitle

% Sisällysluettelo
\newpage
\tableofcontents
\thispagestyle{empty}
\newpage
\setcounter{page}{1}
\parskip=1em \advance\parskip by 0pt plus 2pt
\pagestyle{fancy}
\cfoot{\thepage}

%%%%%%%%%%%%%%% Oleellinen sisältö alkaa%%%%%%%%%%%%%%%
\section{Johdanto}
Työn tavoitteena on toteuttaa kurssikyselyjärjestelmä, jonka avulla opiskelijoilta voidaan kerätä palautetta kursseista. Järjestelmää voidaan käyttää koko tiedekunnassa ja kyselyt sisältävätkin sekä koko tiedekunnan laajuisia kysymyksiä että laitos- ja kurssikohtaisia kysymyksiä. Kysymysten vastaukset voivat olla avoimia tai ne voidaan valita numeroasteikolta tai muusta annetusta arvojoukosta.

Kurssin luennoitsija voi lisätä, poistaa ja muokata kurssikohtaisia kysymyksiä ja laitoksen ja tiedekunnan hallintohenkilöstö niiden kysymyksiä. Samoin uusille kursseille voidaan luoda uusia kyselyitä ja vanhoja voidaan poistaa. Kyselyn muokkaaminen tai poistaminen edellyttää kuitenkin, ettei kysely ole parhaillaan käynnissä.

Laitos ja tiedekunta voivat hakea yhteenvedon kurssin tai kurssien kysymysten tuloksista kuten vastaajamääristä tai tiettyyn kysymykseen saaduista vastauksista. Kurssin pitäjä saa järjestelmästä yksinkertaisen raportin kyselyn tuloksista aina halutessaan.

%\begin{figure}
%   \centering
%   \begin{subfigure}[b]{0.45\textwidth}
%       \includegraphics[height=7.9cm]{../kuvat/amr-grid.png}
%   \end{subfigure}
%   \begin{subfigure}[b]{0.45\textwidth}
%       \includegraphics[height=7.9cm]{../kuvat/amr-nogrid.png}
%   \end{subfigure}
%   \caption{No kuvahan se siinä}\label{fig:enzogrid}
%\end{figure}


\section{Yleiskuva järjestelmästä}
\subsection{Käyttötapauskaavio}
Käyttötapauskaavio näyttää järjestelmän sidosryhmät ja miten ne liittyvät järjestelmään. Tällä kurssilla kehitettävissä järjestelmissä sidosryhminä ovat enimmäkseen järjestelmän käyttäjäryhmät. Muita mahdollisia sidosryhmiä olisivat esimerkiksi toiset järjestelmät. Harjoitustyössä käyttäjäryhmien tulisi osaltaan näkyä kyseisille ryhmille suunnattuina toiminnallisuuksia.

Käyttötapauskaaviossa käyttötapaukset kuvataan järjestelmän sisällä ja sidosryhmien yhteydet järjestelmään esitetään yhteyksinä käyttötapauksiin. Kaavion kuvaa järjestelmästä voisi kuvailla esimerkiksi “Tällaisia asioita järjestelmässä voi tehdä, ja nämä ryhmät tekevät näitä asioita”.

Kaavion lisäksi käyttäjäryhmät sekä tärkeimmät käyttötapaukset kuvaillaan erikseen myös tekstinä.

\section{Järjestelmän tietosisältö}

\section{Relaatiotietokantakaavio}

\section{Järjestelmän yleisrakenne}

\section{Käyttöliittymä ja järjestelmän komponentit}

\section{Asennustiedot}

\section{Käynnistys- ja käyttöohje}



%%%%% Sisältö loppuu, lähdeluettelo %%%%%
\bibliographystyle{plain}
\small
\bibliography{lahteet}

\appendix
%\newpage
\section{Tärkeä liite}
Lorem ipsum.
\newpage



\end{document}
